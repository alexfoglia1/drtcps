%%%%%%%%%%%%%%%%%%%%%%%%%%%%%%%%%%%%%%%%%
% Beamer Presentation
% LaTeX Template
% Version 1.0 (10/11/12)
%
% This template has been downloaded from:
% http://www.LaTeXTemplates.com
%
% License:
% CC BY-NC-SA 3.0 (http://creativecommons.org/licenses/by-nc-sa/3.0/)
%
%%%%%%%%%%%%%%%%%%%%%%%%%%%%%%%%%%%%%%%%%

%----------------------------------------------------------------------------------------
%   PACKAGES AND THEMES
%----------------------------------------------------------------------------------------

\documentclass{beamer}

\mode<presentation> {

% The Beamer class comes with a number of default slide themes
% which change the colors and layouts of slides. Below this is a list
% of all the themes, uncomment each in turn to see what they look like.

%\usetheme{default}
%\usetheme{AnnArbor}
%\usetheme{Antibes}
%\usetheme{Bergen}
%\usetheme{Berkeley}
%\usetheme{Berlin}
%\usetheme{Boadilla}
%\usetheme{CambridgeUS}
%\usetheme{Copenhagen}
%\usetheme{Darmstadt}
%\usetheme{Dresden}
%\usetheme{Frankfurt}
%\usetheme{Goettingen}
%\usetheme{Hannover}
%\usetheme{Ilmenau}
%\usetheme{JuanLesPins}
%\usetheme{Luebeck}
\usetheme{Madrid}
%\usetheme{Malmoe}
%\usetheme{Marburg}
%\usetheme{Montpellier}
%\usetheme{PaloAlto}
%\usetheme{Pittsburgh}
%\usetheme{Rochester}
%\usetheme{Singapore}
%\usetheme{Szeged}
%\usetheme{Warsaw}

% As well as themes, the Beamer class has a number of color themes
% for any slide theme. Uncomment each of these in turn to see how it
% changes the colors of your current slide theme.

%\usecolortheme{albatross}
%\usecolortheme{beaver}
%\usecolortheme{beetle}
%\usecolortheme{crane}
%\usecolortheme{dolphin}
%\usecolortheme{dove}
%\usecolortheme{fly}
%\usecolortheme{lily}
%\usecolortheme{orchid}
%\usecolortheme{rose}
%\usecolortheme{seagull}
%\usecolortheme{seahorse}
%\usecolortheme{whale}
%\usecolortheme{wolverine}

%\setbeamertemplate{footline} % To remove the footer line in all slides uncomment this line
%\setbeamertemplate{footline}[page number] % To replace the footer line in all slides with a simple slide count uncomment this line

%\setbeamertemplate{navigation symbols}{} % To remove the navigation symbols from the bottom of all slides uncomment this line
}

\usepackage{graphicx} % Allows including images
\usepackage{booktabs} % Allows the use of \toprule, \midrule and \bottomrule in tables

%----------------------------------------------------------------------------------------
%   TITLE PAGE
%----------------------------------------------------------------------------------------

\title[Kilobot Platooning - General Requirements]{Lab. Distributed Real Time Cyberphysical Systems\\Kilobot Plaotoning - General Requirements Definition} % The short title appears at the bottom of every slide, the full title is only on the title page

\author{Francesco Terrosi \\Alex Foglia} % Your name
\institute[UNIFI] % Your institution as it will appear on the bottom of every slide, may be shorthand to save space
{
Universit\`a di Firenze \\ % Your institution for the title page
}
\date{\today} % Date, can be changed to a custom date

\begin{document}

\begin{frame}
\titlepage % Print the title page as the first slide
\end{frame}


\begin{frame}
\frametitle{Viewpoints}
\begin{itemize}
	\item Architecture:
	\begin{itemize}
		\item Environment
		\item SoS Organization
	\end{itemize}
	\item Communication:
	\begin{itemize}
		\item CS-Level
		\item SoS-Level
	\end{itemize}
	\item Emergence
	\item Dinamicity
	\item Time
\end{itemize}
\end{frame}

\begin{frame}
\frametitle{Viewpoint Architecture}
Environment:
	\begin{enumerate}[align=left]
		\item [AE-1:] The kilobots shall operate on a whiteboard.
		\item [AE-2:] An obstacle shall be located in the middle of the whiteboard
	\end{enumerate}
\end{frame}

\begin{frame}
\frametitle{Viewpoint Architecture}
SoS Organization:

\begin{enumerate}[align=left]
	\item[ASoS-1:] The SoS is composed by N kilobots
	\item  [ASoS-2:] The SoS is also composed by a controller that loads the program in kilobots memory
	\item [ASoS-3:] The SoS shall be a platooning among the kilobots composing the SoS
	\item [ASoS-4:] When the SoS starts, kilobots shall be positioned in a straight line, at a distance of D cm
	\item [ASoS-5:] When the SoS operates, distance betweeen kilobots shall be mantained approximately D cm
	\item [ASoS-6:] The leader is decided before execution
	\item [ASoS-7:] All kilobots shall know leader's ID
\end{enumerate}
\end{frame}

\begin{frame}
\frametitle{Viewpoint Communication}
CS-Level:

\begin{enumerate}[align=left]
	\item[CCS-1:] Each kilobot has a RUMI to exchange messages among each other
	\item[CCS-2:] Each kilobot has a RUMI to communicate with the controller
	\item[CCS-3:] Controller has a RUMI to communicate with kilobots
	\item[CCS-4:] Each kilobot has a RUPI to estimate distances
\end{enumerate}
\end{frame}

\begin{frame}
\frametitle{Viewpoint Communication}
SoS-Level:

\begin{enumerate}[align=left]
	\item[CSoS-1:] Each kilobot shall use its RUMI to exchange informations about:
	\begin{itemize}
		\item Direction
		\item When it is Joining the platoon
		\item When it is Leaving the platoon
	\end{itemize}
	\item[CSoS-2:] When the SoS starts, each kilobot notifies to his adjacent follower he is his leader, by transmitting a message
	\item[CSoS-3:] Each kilobot has a RUPI to estimate distance betweeen sender and receiver using signal power
\end{enumerate}
\end{frame}

\begin{frame}
\frametitle{Viewpoint Emergence}
\begin{enumerate}[align=left]
	\item[E-1:] The interaction of mutiple kilobots shall originate a unique platoon
\end{enumerate}
\end{frame}

\begin{frame}
\frametitle{Viewpoint Dinamicity}
\begin{enumerate}[align=left]
	\item[D-1:] The platoon shall allow any kilobot to enter the platoon
	\item[D-2:] The platoon shall be composed of at least 2 kilobots
	\item[D-3:] The introduction of a kilobot in the platoon shall be allowed only at its tail
	\item[D-4:] The platoon shall allow only last kilobot to leave the platoon
\end{enumerate}
\end{frame}

\begin{frame}
\frametitle{Viewpoint Time}
\begin{enumerate}[align=left]
	\item[T-1:] Kilobot shall measure time according to a local clock
	\item[T-2:] Timely-related events shall be triggered by message exchange 
	\item[T-3:] When a kilobot starts shall prepare motors for M ms
\end{enumerate}
\end{frame}

\end{document}
              